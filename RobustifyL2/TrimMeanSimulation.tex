\documentclass{article}\usepackage[]{graphicx}\usepackage[]{color}
%% maxwidth is the original width if it is less than linewidth
%% otherwise use linewidth (to make sure the graphics do not exceed the margin)
\makeatletter
\def\maxwidth{ %
  \ifdim\Gin@nat@width>\linewidth
    \linewidth
  \else
    \Gin@nat@width
  \fi
}
\makeatother

\definecolor{fgcolor}{rgb}{0.345, 0.345, 0.345}
\newcommand{\hlnum}[1]{\textcolor[rgb]{0.686,0.059,0.569}{#1}}%
\newcommand{\hlstr}[1]{\textcolor[rgb]{0.192,0.494,0.8}{#1}}%
\newcommand{\hlcom}[1]{\textcolor[rgb]{0.678,0.584,0.686}{\textit{#1}}}%
\newcommand{\hlopt}[1]{\textcolor[rgb]{0,0,0}{#1}}%
\newcommand{\hlstd}[1]{\textcolor[rgb]{0.345,0.345,0.345}{#1}}%
\newcommand{\hlkwa}[1]{\textcolor[rgb]{0.161,0.373,0.58}{\textbf{#1}}}%
\newcommand{\hlkwb}[1]{\textcolor[rgb]{0.69,0.353,0.396}{#1}}%
\newcommand{\hlkwc}[1]{\textcolor[rgb]{0.333,0.667,0.333}{#1}}%
\newcommand{\hlkwd}[1]{\textcolor[rgb]{0.737,0.353,0.396}{\textbf{#1}}}%

\usepackage{framed}
\makeatletter
\newenvironment{kframe}{%
 \def\at@end@of@kframe{}%
 \ifinner\ifhmode%
  \def\at@end@of@kframe{\end{minipage}}%
  \begin{minipage}{\columnwidth}%
 \fi\fi%
 \def\FrameCommand##1{\hskip\@totalleftmargin \hskip-\fboxsep
 \colorbox{shadecolor}{##1}\hskip-\fboxsep
     % There is no \\@totalrightmargin, so:
     \hskip-\linewidth \hskip-\@totalleftmargin \hskip\columnwidth}%
 \MakeFramed {\advance\hsize-\width
   \@totalleftmargin\z@ \linewidth\hsize
   \@setminipage}}%
 {\par\unskip\endMakeFramed%
 \at@end@of@kframe}
\makeatother

\definecolor{shadecolor}{rgb}{.97, .97, .97}
\definecolor{messagecolor}{rgb}{0, 0, 0}
\definecolor{warningcolor}{rgb}{1, 0, 1}
\definecolor{errorcolor}{rgb}{1, 0, 0}
\newenvironment{knitrout}{}{} % an empty environment to be redefined in TeX

\usepackage{alltt}
\usepackage{amsmath,amsfonts,bm,fullpage}
\usepackage{natbib}
\bibliographystyle{abbrvnat}
\newcommand{\ProjMean}{{\widehat{\bm S}_{2}}}
\newcommand{\R}{{\mathbb{R}}}
\IfFileExists{upquote.sty}{\usepackage{upquote}}{}
\begin{document}

\begin{center}
\Large{\bf Robustifying the Projected Mean}
\end{center}
\normalsize
This is the literature I have found methods to robustify the $L_2$ estimator for various data types.  Specifically I focus on the trimmed and winsorized means.



 
\section{Trimmed Means}

Assume the sample of size $n$, $x_1,\dots,x_n$, has empirical distribution function $F_n$.  The sample $\alpha$-trimmed mean, according to \cite{huber2009} page 10, is
\[
\bar{X}_\alpha=\frac{1}{1-2\alpha}\int_{\alpha}^{1-\alpha}F_n^{-1}(t)dt.
\]

The following is taken from Section 4 of \cite{laha2011}. In the circular context, suppose $\theta$ is a circular random variable with p.d.f.~$f(\theta)$ and $0<\gamma\leq 0.5$ is fixed.  Let $\alpha,\beta$ be two points on the unit circle satisfying
\[
\int_{\beta}^\alpha f(\theta)d\theta=1-2\gamma.
\] 
The circular $\gamma$-trimmed mean is
\[
\mu_\gamma=\text{arg}\left[\frac{1}{1-2\gamma}\int^{\alpha}_\beta\exp(\imath\theta)f(\theta)d\theta\right].
\]
In their Theorem 4.1 they show it is SB-robust for the circular von Mises distribution.  In \cite{laha2013} they show it is SB-robust for the wrapped normal distribution too.  Remember SB-robustness is always with respect to some dispersion measure.




\section{Weighted Means}

According to \cite{huber2009} Section 11.2.2 the current best possible break down for a $d$-dimensional affine equivilant estimator is
\[
\epsilon^*=\frac{n-2d+1}{2n-2d+1}.
\]
So for us, this would be $(n-5)/(2n-5)$.  This break-down is achieved by the weighted average of the points $\bm x_i$ from $\bm X$ with weights $w_i=w(r_i)$ where
\[
r_i=\sup_{\bm u}\frac{\bm u^\top\bm x_i-\text{MED}(\bm u^\top\bm X)}{\text{MAD}(\bm u^\top\bm X)}
\]
and $w(r)$ is a strictly positive, decreasing fuction of $r\geq 0$, with $w(r)r$ bounded.  I think $r_i$ can be replaced with any one-dimensional projection of the outlyingness of the $\bm x_i$.  For example, $H_n$ or the distance the mean moves when $\bm R_i$ is removed.







\section{Winsorized Means}
The winsorized mean (in my mind) coincides with the multidimensional Huber estimator.  From the intervals Appendix, the infulece function of $\ProjMean$ for $\bm R(r,\bm u)$ distributed according to a directionally symmetric distribution on $SO(3)$ $F$ is
\[
\text{IF}_2(\bm R_i,F)=\frac{3}{1+2E[\cos(r)]}\sin(r)\bm u.
\]

Thus the gross error sensitivity is
\[
\|\text{IF}_2(\bm R_i,F)\|=\|\frac{3}{1+2E[\cos(r)]}\sin(r)\bm u\|=\left|\frac{3\sin(r)}{1+2E[\cos(r)]}\right|=\frac{3\sin(|r|)}{1+2E[\cos(r)]}
\]







%\bibliographystyle{plain}
\bibliography{../OutlierDetection/RobustRefs}
\end{document}
